\documentclass[11pt]{article}
\usepackage{euscript}

\usepackage{amsmath}
\usepackage{amsthm}
\usepackage{amssymb}
\usepackage{epsfig}
\usepackage{xspace}
\usepackage{color}
\usepackage{url}
\usepackage{quiver}

%%%%%%%  For drawing trees  %%%%%%%%%
\usepackage{tikz}
\usetikzlibrary{calc, shapes, backgrounds}

%%%%%%%%%%%%%%%%%%%%%%%%%%%%%%%%%
\setlength{\textheight}{9in}
\setlength{\topmargin}{-0.600in}
\setlength{\headheight}{0.2in}
\setlength{\headsep}{0.250in}
\setlength{\footskip}{0.5in}
\flushbottom
\setlength{\textwidth}{6.5in}
\setlength{\oddsidemargin}{0in}
\setlength{\evensidemargin}{0in}
\setlength{\columnsep}{2pc}
\setlength{\parindent}{1em}
%%%%%%%%%%%%%%%%%%%%%%%%%%%%%%%%%
\theoremstyle{definition}
\newtheorem{definition}{Definition}[section]
\newtheorem{example}{Example}[section]
\newtheorem{theorem}{Theorem}[section]
\newtheorem{corollary}{Corollary}[theorem]
\newtheorem{lemma}[theorem]{Lemma}

\newcommand{\eps}{\varepsilon}

\renewcommand{\c}[1]{\ensuremath{\EuScript{#1}}}
\renewcommand{\b}[1]{\ensuremath{\mathbb{#1}}}
\newcommand{\s}[1]{\textsf{#1}}

\newcommand{\E}{\textbf{\textsf{E}}}
\renewcommand{\Pr}{\textbf{\textsf{Pr}}}
\DeclareMathOperator{\Ima}{im}
\DeclareMathOperator{\coker}{coker}
\DeclareMathOperator{\Tor}{Tor}
\DeclareMathOperator{\id}{id}

\newcommand{\Z}{\mathbb{Z}}

\title{Tensors and Homology}
\author{Annie Giokas, Andrew Campbell}

\begin{document}
\maketitle
\section{Exact Sequences}

A sequence of $A-$modules and $A-$homomorphisms 
$$\cdots\rightarrow M_{i-1}\rightarrow^{f_{i}} M_{i}\rightarrow^{f_{i+1}} M_{i+1}\rightarrow \cdots$$ is \textit{exact at} $M_i$ if  $Im(f_i)=\ker(f_{i+1})$.  The sequence is \textit{exact} if it is exact at each $M_i$.

\begin{lemma}
 Properties of exact sequences:\\
\begin{itemize}
    \item $0\rightarrow M' \rightarrow^{f}M$ is exact if and only if $f$ is injective.
    \item $M\rightarrow^{g}M''\rightarrow 0$ is exact if and only if $g$ is surjective.  
    \item $0\rightarrow M' \rightarrow^{f}M\rightarrow^{g}M''\rightarrow 0$ is exact if and only if $f$ is injective, $g$ is surjective and $g$ induces an isomorphism of $\coker(f)=M/f(M')$ onto $M''$.
 \end{itemize}
\end{lemma}
\proof
$0\rightarrow M' \rightarrow^{f}M$ is exact if and only if we have $\Ima(0)=\ker(f)$ by definition so that means that $\ker(f)=0$ and this is true if and only if $f$ is injective. \\

$M\rightarrow^{g}M''\rightarrow 0$ is exact then $\Ima(g) = \ker(0)$ and since $\ker(0)=M'$ we have $\Ima(g) = M'$ and this is true if and only if $g$ is surjective. 

\qed

A sequence of the type $0\rightarrow M' \rightarrow^{f}M\rightarrow^{g}M''\rightarrow 0$ is called a \textit{short exact sequence}. Note  that any long exact sequence $\cdots\rightarrow M_{i-1}\rightarrow^{f_{i}} M_{i}\rightarrow^{f_{i+1}} M_{i+1}\rightarrow \cdots$ can be split up into short exact sequences. If  $N_i=\Ima(f_i)=\ker(f_{i+1})$ we have $0\rightarrow N_i\rightarrow M_i\rightarrow N_{i+1}\rightarrow 0$ for all $i$.

\begin{example}
A simple and very general example of a short exact sequence is generated by an arbitrary homomorphism. Let $f:A\rightarrow B$ be a homomorphism of some kind (be it between groups, rings or modules). Then it gives rise to this exact sequence:
$$0\rightarrow \ker{f}\rightarrow f \rightarrow \Ima{f}\rightarrow 0 $$
\end{example}
\section{Tensor Products}

In general, tensors can be regarded as many types of objects depending on the context but for it to be relevant to our class we are going to treat it as a ring module and then as a functor. The word \textit{functor} comes from Category theory but hopefully it will make sense with the way we present it here in regards to the construction of the Tor functor for the purpose of handling problems associated with sequences better. To vaguely motivate introducing tensor products as a concept, we can think about the convenience of being able to take bilinear maps and to turn them into linear maps (in this case we are going to be talking about R-module homomorphisms).

Let $A$ be a ring and let $M,N,P$ be $A$-modules.
We will be constructing a new A-module $T$ called the \textit{tensor product} of $M$ and $N$ with the property that A-bilinear mappings $M\times N\rightarrow P$ are in a natural correspondence which is bijective with the A-linear mappings $T\rightarrow P$ for any $P$ A-module. 
$\{ f:M\times N\rightarrow P \text{A-linear}\}\leftrightarrow \{g:T\rightarrow P \text{ A-linear}\}$.

\begin{definition}[$A$-bilinearity]
$A$-bilinear means that $F:M\times N\rightarrow P$ it satisfies these conditions:
\begin{itemize}
    \item $f(\lambda m , n)=f(m,\lambda n)=\lambda f(m,n)$
    \item $f(m,n_1+n_2)=f(m,n_1)+f(m,n_1)$
    \item $f(m_1+m_2,n)=f(m_1,n)+f(m_2,n)$
\end{itemize}
where $\lambda\in A$, $m_1,m_2,n\in M$ and $n_1,n_2,n\in N$.
\end{definition}

\begin{definition}[Tensor product of modules]
A \textit{tensor product} of two $A$-modules $M$ and $N$ is an $A$-module denoted by $M\otimes N$ such that there exists an $A$-bilinear mapping $g:M\times N \rightarrow M\otimes N$ with the following universal property:\\
given any $A$-module $P$ and any $A$-bilinear mapping $f:M\times N\rightarrow P$, there exists a unique $A$-linear mapping (be careful, $A$-linear and not $A$-bilinear!) $f':M\otimes N \rightarrow P$ such that the diagram below commutes and we have $f=f'\circ g$.
(Another way of stating this is that every bilinear function on $M\times N$ factors through $M\otimes N$)
\[\begin{tikzcd}
	&& {M\otimes N} \\
	& {} \\
	{M\times N} &&&& P
	\arrow["{\exists g}", from=3-1, to=1-3]
	\arrow[from=3-1, to=3-5]
	\arrow["{f'}", dashed, from=1-3, to=3-5]
\end{tikzcd}\]
\end{definition}





\begin{theorem}[Construction of the tensor product]
For any $A$-modules $M$ and $N$ there exists a pair $(M\otimes_A N,g)$ consisting of an $A$-module and a mapping from $M\times N\rightarrow M\otimes N$ which has the universal property described above. Additionally, the tensor product is unique up to isomorphism. In other words, if $(T,g)$ and $(T',g')$ are two pairs with this property, then there exists a unique isomorphism $J:T\rightarrow T'$ such that $j\circ g=g'$.
\end{theorem}
\proof 

\textit{Existence:}
    Let $C$ be a free $A$ module $A^{M\times N}$. The elements of $C$ are linear combinations of elements of $M\times N$ with coefficients in $A$, so they are of the form $\sum_{i=1}^{n}a_i(x_i,y_i)$ where $a_i\in A,x_i\in M,y_i\in N$. Let $D$ be a submodule of $C$ generated by all elements of $C$ of the following types:
    $$(x+x',y)-(x,y)-(x',y) $$
    $$(x,y+y')-(x,y)-(x,y') $$
    $$(ax,y)-a(x,y) $$
    $$(x,ay)-a(x,y) $$
    
    Let us define $T:=C/D$. For each basis element $(x,y)$ of $C$, let $x\otimes y$ denote its image in $T$. Then $T$ is generated by the elements of the form $x\otimes y$, and from our definitions we have 
    $$(x+x')\otimes y=x\otimes y+x'\otimes y $$
    $$x\otimes(y+y')=x\otimes y+x\otimes y' $$
    $$(ax)\otimes y=x\otimes (ay)=a(x\otimes y) $$
    
    Equivalently, the mapping $g:M\times N\rightarrow T$ defined by $g(x,y)=x\otimes y$ is $A$-bilinear.
    Any map $f$ of $M\times N$ into an $A$-module $P$ extends by linearity to an $A$-module homomorphism $\overline{f}:C\rightarrow P$. Suppose that $f$ is $A$-bilinear. Then by using he definitions, $\overline{f}$ vanishes on all the generators of $D$, hence on the whole of $D$. This means that it induces a well-defined $A$-homomorphism $f'$ of $T$ into $P$ such that $f'(x\otimes y)=f(x,y)$. The mapping $f'$ is uniquely defined by this condition, therefore the pair $(T,g)$ satisfy the conditions of the proposition. \\
   
\textit{Uniqueness:}
    
    Let $(T,g)$ and $(T',g')$ be the tensor products of $M$ and $N$. We will treat $T'$ as the $A-module$ and $g'$ as the $A$-bilinear mapping from $M\times N$ to $T'$, then it is easy to see that that by the universal property of $(T,g)$, we get a unique $j:T\rightarrow T'$ such that $g'=j\circ g$. Then we do the same thing, except we switch the roles of $(T,f)$ and $(T',f')$, we get a unique $j': T'\rightarrow T$ such that $g=j'\circ g'$. From the composition equalities, we can conclude that $j\circ j'$ and $j'\circ j$ must be the identity, therefore $j=j'$ and is an isomorphism. This is why we say that tensor products are \textit{unique up to isomorphism}.
 $\qed$\\
 
 Of course, we can use induction to generalize tensor products, where instead of working with $A$-bilinear mappings, we work with $A$-multilinear mappings. Below are some of the more commonly used tensor product isomorphisms that can be proved by diagram chasing methods. 
 
 \begin{lemma}
 Let $M,N,P$ be $A$-modules. Then there exist unique isomorphisms:
 \begin{enumerate}
     \item $M\otimes N\rightarrow N\otimes M$
     \item $(M\otimes N)\otimes P\rightarrow M\otimes(N\otimes P)\rightarrow M\otimes N\otimes P$
     \item $(M\oplus N)\otimes P\rightarrow (M\otimes P)\oplus(M\otimes P)$
     \item $A\otimes M\rightarrow M$
 \end{enumerate}
 \end{lemma}
Now we are going to connect the concept of tensors to exact sequences we introduced in the beginning. We are going to be \textit{tensoring} the sequence. 

First, we must introduce the concept of left and right exactness, which tells us which part of the sequence is preserved to be exact by a functor. 
\begin{definition}
If we have a functor, say $F$ and a a short exact sequence $0\rightarrow A\rightarrow B\rightarrow C\rightarrow 0$, then the functor $F$ is called \textit{right exact} if the new sequence $F(A)\rightarrow F(B)\rightarrow F(C)\rightarrow 0$ is exact. It is \textit{left exact} if $0\rightarrow F(A)\rightarrow F(B)\rightarrow F(C)$ is exact.
\end{definition}

It turns out that tensor products are right exact, but not necessarily left exact. 
\begin{theorem}[Right exactness of the tensor product]
Let $M'\rightarrow^{f}M\rightarrow^{g} M''\rightarrow 0$ be an exact sequence of $A$-moudules and homomorphisms, and let $N$ be any $A$-module. Then the sequence
$$M'\otimes N\rightarrow^{f\otimes 1}\rightarrow M\otimes N\rightarrow^{g\otimes 1}M''\otimes N\rightarrow 0 $$
is exact. Note that $1$ here denotes the identity mapping on $N$. 
\end{theorem}

\begin{example}
Let us consider the exact sequence $0\rightarrow \Z\xrightarrow{f}\rightarrow \Z$ where we have $\Z$-modules and $(x)=2x$ for all $x\in \Z$. If we tensor with $N=\Z/2\Z$, we get the sequence 
$$0\rightarrow \Z\otimes N\xrightarrow{f\otimes 1}\Z\otimes N $$
which is not exact because for any $x\otimes y\in \Z\otimes N$ we hav e
$$(f\otimes 1)(x\otimes y)=2x\otimes y=x\otimes 2y=x\otimes 0=0 $$
This makes $f\otimes 1$ the zero mapping, but it is clear that $\Z\otimes N\neq 0$. This shows that tensor products are sometimes not left exact. 
\end{example}

\begin{lemma}
$(\Z\backslash m\Z)\otimes_{\Z}(\Z\backslash m \Z)=0$ if $m,n$ are coprime.
\proof
If $m,n$ are co-prime then we can find $r,s$ such that $mr+ns=1$. Let us take an arbitrary element $x\times y$ of $(\Z\backslash m\Z)\otimes_{\Z}(\Z\backslash m \Z)$. Then 
$$x\otimes y= 1\cdot x\otimes y=(mr+ns)(x\otimes y)$$
\end{lemma}


\section{Tor Functor}

\begin{definition}[Chain complex]
A \textit{chain complex} is a sequence of modules 
$$\cdots\rightarrow A_{n+2}\rightarrow^{\partial_{n+2}} A_{n+1}\rightarrow^{\partial_{n+1}} A_{n}\rightarrow^{\partial_n} \cdots $$
where every $\partial_{i}\circ \partial_{i+1}=0$
\end{definition}
As a note, an exact sequence is essentially a chain complex, where its homology groups are all zero. Informally, Homology groups represent topological differences within a space. More percisely, for topoligcal space $X$, $H_k(X)$ is the $k^{th}$ homology group of $X$ and $H_k(X)$ corresponds to the number of $k$-dimensional holes in $X$. For example: for the torus ($T^2$) $H_0(T^2)=\mathbb{Z}$ since the torus is connected.  



$H_i(F_{*})=\frac{\ker{\partial_i}}{\Ima{\partial_{i+1}}}$\\

Note: $\partial_{i}\circ \partial_{i+1}=0$ iff $\Ima{\delta_{i+1}}\subseteq\ker{\delta_i}$\\



\begin{definition}[Free resolution]
Let $M$ be an $R$-module.
Let $M$ be an $R$-module of a ring $R$. A \textit{free resolution} of $M$ is a complex of free $R$-modules
$$F_{*}=\cdots\rightarrow F_3\rightarrow ^{\partial_3}F_2\rightarrow^{\partial_2}F_1\rightarrow^{\partial_1}\rightarrow F_0\rightarrow 0 $$
where $F_i$ are free $R$-modules such that
\begin{itemize}
    \item $H_0(F_{*})=M$
    \item $H_i(F_{*})=0$
\end{itemize}
\end{definition}

\begin{lemma}
These statements are equivalent:
\begin{enumerate}
    \item $H_0(F)=M$
    \item $\frac{\ker\partial_0}{\Ima{\partial_1}}=M$
    \item $\frac{F_0}{\Ima{\partial_1}}=M$
    \item $\coker\partial_1 =M$
    \item $F_1\rightarrow^{\partial_1}F_0\rightarrow M\rightarrow 0$ is exact
\end{enumerate}
\end{lemma}
We can think of $F_{*}$ as an approximation (possibly infinite) of $M$ by free $R$-modules.\\ 

Now we will finally be introducing the Tor functor and the reason it is needed is because of the inadequecy of tensor products. As we have shown above, tensor product as a functor is Right exact but not Left exact and to "compensate" for that we use Tor.

\begin{definition}[Tor functor]
For $R$-modules $M$ and $N$, $\Tor_i^R(M,N)$ is an $R$-module defined by the process:
Take some free resolution $F_{*}$ of $M$. 
    $$F_{*}=\cdots\rightarrow F_3\rightarrow ^{\partial_3}F_2\rightarrow^{\partial_2}F_1\rightarrow^{\partial_1}\rightarrow F_0\rightarrow 0 $$
Next tensor $F_{*}$ with $N$
$$F_{*}\otimes N=\cdots\rightarrow F_3\otimes N\rightarrow ^{\partial_3}F_2 \otimes N\rightarrow^{\partial_2}F_1\otimes N\rightarrow^{\partial_1}\rightarrow F_0\otimes N\rightarrow 0 $$
then we define
    $$\Tor_i^R(M,N):=H_i(F_{*}\otimes_R N)=\frac{\ker(\partial_i\otimes \id)}{\Ima(\partial_{i+1}\otimes \id)}$$

\end{definition}

\begin{example}
If we take $R$ to be $\Z$ and treat $M$ as a $\Z$-module then we will actually have an abelian group. As a note the definition of complexes does work with abelian groups and modules in general. We can compute the $Tor_i^{\Z}(\Z/ (n))$ $\forall i$ where $n\geq 2$.
%First we can take the complex 



\end{example}









\section{Universal Coefficient Theorem}
The Universal Coefficient Theorem shows that there exists a relationship between homology groups with different coefficients. More precisely, it shows that there is a relationship between $\mathbb{Z}$ coefficients and arbitrary coefficients in the context of homology. This is useful because it allows us to move between different homology groups with different coefficients using Tor. Below is the theorem for the homology.
\begin{theorem}
If $C$ is a chain complex of free abelian groups and $G$ is an $R-Module$, then there exists a short exact sequence, for all $n$ and $G$: $$0\rightarrow H_n(C)\otimes G \rightarrow H_n(C;G)\rightarrow Tor_1(H_{n-1}(C),G)\rightarrow 0$$ and this sequence splits. 
\end{theorem}

Before providing the proof, there are a few definitions and properties to be familiar with: 
\begin{definition}[$coKer$]
For $f:A\rightarrow B$ a group homorophism $coKer(f)=B/Im(f)$
\end{definition}
\begin{definition}[Projective]
For $P$ an $R-Module$, $P$ is {projective} if for $L,N,M$ $R-Modules$,  $0\rightarrow L\xrightarrow[]{\psi} M \xrightarrow[]{\phi} N\rightarrow 0$ is exact, then $0\rightarrow Hom_R(L,P)\xrightarrow[]{\psi'} Hom_R(M,P) \xrightarrow[]{\phi'} Hom_R(N,P)\rightarrow 0$ is also exact.
\end{definition}
\begin{definition}[Projective Resolution]
Let $A$ be an $R-Module$. A projective resolution of $A$ is an exact sequence: 
    $$...\rightarrow P_n\xrightarrow[]{d_n} P_{n-1}\rightarrow...\xrightarrow[]{d_1}P_0\xrightarrow[]{\epsilon} A\rightarrow 0$$ 
    where all $P_i$ are projective.
\end{definition}
Below are some useful lemmas used during the proof of the theorem:
\begin{lemma}
For $A,B$ right $R-Modules$ and $C,D$ left $R-modules$, there exist unique isomorphisms:
    $$(A\oplus B)\otimes C\simeq (A\otimes C)\oplus (B\otimes C)$$
    and similarly:
    $$A\otimes(C\oplus D)\simeq (A\otimes C)\oplus (A\otimes D)$$
\end{lemma}
\begin{lemma}
Let $0\rightarrow L\xrightarrow[]{\psi} M \xrightarrow[]{\phi} N\rightarrow 0$ be a short exact sequences of chain complexes. We want to show that we can stretch this sequence into a long exact sequence of homology groups. i.e:
    $$...\rightarrow H_{i+1}(N)\rightarrow H_{i}(L)\rightarrow H_{i}(M)\rightarrow H_{i}(N)\rightarrow H_{i-1}(L)\rightarrow...$$ Now we will define a free projective resolution of $L,M,N$ so we have: 
    $$...\rightarrow P_1\rightarrow P_0\xrightarrow[]{\epsilon} L\rightarrow 0$$
    $$...\rightarrow P'_1\rightarrow P'_0\xrightarrow[]{\epsilon'} M\rightarrow 0$$
    $$...\rightarrow P''_1\rightarrow P''_0\xrightarrow[]{\epsilon''} N\rightarrow 0$$
    Then, if we tensor by the $R-Module$ $D$ we get:
    $$...\rightarrow P_2\otimes D \rightarrow P_1\otimes D\rightarrow P_0\otimes D\rightarrow 0$$
    $$...\rightarrow P'_2\otimes D \rightarrow P'_1\otimes D\rightarrow P'_0\otimes D\rightarrow 0$$
    $$...\rightarrow P''_2\otimes D \rightarrow P''_1\otimes D\rightarrow P''_0\otimes D\rightarrow 0$$
    $$...\rightarrow P_0\otimes D \xrightarrow[]{\psi} P'_0\otimes D\xrightarrow[]{\phi} P''_0\otimes D\rightarrow 0$$
    This is cool, because then we use Tor and get our long exact sequence. we have:
    $$...\rightarrow Tor_2(N,D)\xrightarrow[]{\delta_1}Tor_1(L,D)\xrightarrow[]{\psi_*}Tor_1(M,D)\xrightarrow[]{\phi_*}Tor_1(N,D)\xrightarrow[]{\delta_0}L\otimes D\xrightarrow[]{\psi_*}M\otimes D\xrightarrow[]{\phi_*}N\otimes D\rightarrow 0$$
\end{lemma}
Now for the proof.
\begin{enumerate}
    \item [Proof:]\text{ }\\
    %need to define d_i and L_{i-1}
    Let $D,L,N$ all be $R-Modules$. First we define $\delta_i:Tor_i(N,D)\rightarrow Tor_{i-1}(L,D)$ which is a homomorphism where $[n]\mapsto[l]$ where $n\in ker(d_i)$ and $l\in L_{i-1}$. 
    Then, we let $...\rightarrow C_n\xrightarrow[]{\delta_i}C_{n-1}\rightarrow$ be a chain complex of free abelian groups. We let $B_n=Im(\delta_n)$ and $A_n=ker(\delta_n)\subset C_n$. Also notice that $B_n\subset A_n$. 
    This means that $\delta_n|_{A_n}=0=\delta_n|_{B_n}$. Therefore, we can form the following short exact sequence: 
    $$0\rightarrow A_n\rightarrow C_n\xrightarrow[]{\delta_n} B_{n-1} \rightarrow 0$$
    Since subgroups of free abelian groups are also free abelian groups, we have that \\$C_n\simeq A_n\oplus B_{n-1}$. Next we will tensor our short exact sequence with $G$. We now have: 
    $$0\rightarrow A_n\otimes G\rightarrow C_n\otimes G\xrightarrow[]{\delta_n\otimes 1} B_{n-1}\otimes G \rightarrow 0$$
    We then can apply the Lemma 4.2 to realize that the above sequence is split. Then, using Lemma 4.3, we can identity a long sequence of the homology groups: 
    $$...\rightarrow H_{n}(A;G)\rightarrow H_{n}(C;G)\rightarrow H_{n}(B;G)\rightarrow H_{n-1}(A;G)\rightarrow...$$ 
    However, if we return to the earlier sequence, we observe that the chain complex of $A_n$ only contains the zero homomorphims. This implies that $H_n(A;G)=A_n\otimes G/0=A_n\otimes G$, $\forall n$. This is also the case for the chain complex of $B_n$. Therefore we also have that $H_n(B;G)=B_n\otimes G$. This is every useful since it means that our long sequence is isomorphic to:
    $$...\rightarrow B_n\otimes G\rightarrow A_n\otimes G \rightarrow\rightarrow H_n(C;G)\rightarrow B_{n-1}\otimes G\rightarrow A_{n-1}\otimes G \rightarrow ...$$
    Now, we pick some $b\otimes g\in B_n\otimes G$. But, since we know that $\delta_n \otimes 1$ is surjective, it implies that there exists a $c\otimes g\in C_n\otimes G$ such that $\delta_n\otimes 1(c\otimes g)=b\otimes g$. 
    Additionally, since $B_{n-1}\subset A_{n-1}\otimes G$ and $b\otimes g\in B_n\otimes G$, we know that $b\otimes g\in A_{n-1}\otimes G$. 
    This is useful since we can now define a boundary map $f:B_n\otimes G\rightarrow A_n\otimes G$. 
    Let $i_n$ be the inclusion map $B_n\rightarrow A_n$, then, we can define $f$ as $f=i_n\otimes 1$.
    We relabel $H_n(C;G)$ as $D_n$. Therefore, we can also write $D_{n+1}=A_n\otimes G$ and $D_{n+2}=B_n\otimes G$. Next we will define an extension of $C_{n+1}$ by $C_{n}$ using our inclusion maps from above. Therefore, we have: 
    $$C_n\simeq coker(D_{n+2}\rightarrow D_{n+1})=coker(B_n\otimes G\rightarrow A_n\otimes G)=coker(i_n\otimes 1)$$
    and 
    $$C_{n-1}\simeq coker(D_{n-1}\rightarrow D_{n-2})=coker(B_{n-1}\otimes G\rightarrow A_{n-1}\otimes G)=coker(i_{n-1}\otimes 1)$$
    This is significant because we can now form the short exact sequence:
    $$0\rightarrow coker(i_n\otimes)\rightarrow H_n(C;G)\rightarrow ker(i_{n-1}\otimes 1)\rightarrow 0$$
    where $coker(i_n\otimes 1)=A_n\otimes G/Im(i_n\otimes1)$\\
    Notice that a general group homomorphism $f:A\rightarrow B$, we have an exact short sequence: $$A\xrightarrow[]{f}B\rightarrow coker(f)\rightarrow 0$$
    In this context, we already know that for $B_n\xrightarrow[]{i_n}A_n\rightarrow coker(i_n)\rightarrow 0$, that $coker(i_n)=H_n(C)$. By right exactness of the tensor product we have the following sequences: 
    $$B_n\otimes G\xrightarrow[]{i_n \otimes 1}A_n\otimes G\rightarrow coker(i_n\otimes 1)\rightarrow 0$$
    $$B_n\otimes G\xrightarrow[]{i_n \otimes 1}A_n\otimes G\rightarrow H_n(C)\otimes G\rightarrow 0$$
    This implies, independent of the choice of $B_n$ and $A_n$ that $ coker(i_n\otimes 1)\simeq  H_n(C)\otimes G$
    Now, we want to determine the $ker(i_{n}\otimes 1)$. We will use the free resolution of $H_n(C)$ so we get $$0\rightarrow B_n\xrightarrow[]{i_n}A_n\rightarrow H_n(C)\rightarrow 0$$ 
    Then, when we tensor with $G$ we get:
    $$0\rightarrow B_n\otimes G\xrightarrow[]{i_n}A_n\otimes G\rightarrow H_n(C)\otimes G\rightarrow 0$$
    Now, all we have left is to apply $Tor$. First notice that $Tor^{\mathbb{Z}}_1(H\otimes G)=H_1(B_n\otimes G)=Ker(i_n\otimes 1)$. Therefore we get the following sequunce:
    $$0\rightarrow H_n(C)\otimes G \rightarrow H_n(C;G)\rightarrow Tor_1(H_{n-1}(C),G)\rightarrow 0$$
    Therefore we have completed the construction, all we have left now is to show that the above sequence splits.\\
    $\qed$
\end{enumerate}



\end{document}
