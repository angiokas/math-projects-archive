\documentclass{article}
\usepackage{amsfonts,amssymb,amsthm,amsmath,enumerate,verbatim,mathtools,tikz,bm,mathrsfs,tikz-cd,hyperref,enumitem,amsthm,graphicx}
\usepackage{xcolor}
\usetikzlibrary{arrows}

\begin{document}
\newcommand{\N}{\mathbb{N}}
\newcommand{\Z}{\mathbb{Z}}
\newcommand{\R}{\mathbb{R}}
\newcommand{\Q}{\mathbb{Q}}
\newcommand{\C}{\mathbb{C}}
\newcommand{\K}{\mathbb{K}}
\newcommand{\F}{\mathbb{F}}

\newcommand{\g}{\mathfrak{g}}

\newcommand{\B}{\mathcal{B}}
\newcommand{\T}{\mathcal{T}}
\newcommand{\U}{\mathcal{U}}
\newcommand{\D}{\mathcal{D}}
\newcommand{\J}{\mathcal{J}}


\title{Exploring finite representations of $sl(2,\C)$}
\author{Annie Giokas}
\date{11/11/2020}
\maketitle
%\usepackage[dvips]{color}
\pagestyle{empty} 
\setlength{\topmargin}{-.4in} 
\setlength{\textheight}{9.5in} 
\setlength{\textwidth}{7in} 
\setlength{\oddsidemargin}{-.2in} 
\setlength{\evensidemargin}{-.2in}

\newenvironment{topic}[1]{\begin{trivlist}\item[]{\bf #1:}}{\end{trivlist}}

\renewcommand{\qed}{\hfill\blacksquare}
\newcommand{\qedwhite}{\hfill \ensuremath{\Box}}

\newcommand*\interior[1]{\mathring{#1}}
\newcommand*\closure[1]{\overline{#1}} % (or \bar{#1})
\newcommand*\boundary[1]{\partial {#1}}

\newtheorem{theorem}{Theorem}[section]
\newtheorem{corollary}{Corollary}[theorem]
\newtheorem{lemma}[theorem]{Lemma}
\theoremstyle{definition}
\newtheorem{definition}{Definition}[section]

\section{Lie algebras and its representations}
\begin{definition}A \textcolor{red}{Lie algebra} is a vector space over a field $k$ with a bilinear product $[x,y]$ such that the product satisfies the next two conditions:
\begin{itemize}
\item Alternativity:\\
$[x,x]=0$ for any $x\in \g$

\item Jacobi identity:\\
$[[x,y],z]+[[y,z],x]+[[z,x],y]=0$ for any $x,y,z\in \g$
\end{itemize}
\end{definition}
*Alternativity condition is equivalent to $[x,y]=-[y,x]$ for any $x,y\in \g$

For any algebra $\g$ we have a linear map $ad:\g\rightarrow End_k\g$ given by $$(ad\; x)(y)=[x,y]$$

\begin{definition} Let $V$ be a vector space over field $\K$. A \textcolor{red}{representation} of Lie algebra $\g$ on $V$ is a homomorphism of Lie algebras $\pi:\g\rightarrow(End_{\K}V)^{k}$ which we can simply write as 
$$\pi:\g\rightarrow End_{\K}V$$
The way the bracket is defined in $End_{\K}V$ makes $\pi$ k-linear and satisfies:
$$\pi([X,Y])=\pi(X)\pi(Y)-\pi(Y)\pi(X) $$

\end{definition}
\section{$\mathfrak{sl}(2,\C)$}
$\mathfrak{sl}(n,\F)$ is the special linear Lie algebra of order n over a field $\F$. It is comprised of $n\times n$ matrices with trace zero and with the Lie bracket defined like so:
$$[X,Y]=XY-YX $$
We are going to be talking about the scenario when $n=2$ and $\F=\C$. We can describe elements of $\mathfrak{sl}(2,\C)$ explicitly like so:
$$\mathfrak{sl}(2,\C)=\{\begin{pmatrix}
a & b \\
c & -a 
\end{pmatrix}| a,b,c\in \C\}$$

It is a 3 dimensional complex Lie algebra which has the basis:

\begin{align*}
&
h=\begin{pmatrix}
1 & 0\\
0 & -1 
\end{pmatrix} 
&
e=\begin{pmatrix}
0 & 1\\
0 & 0 
\end{pmatrix}
&
&
f=\begin{pmatrix}
0 & 0\\
1 & 0 
\end{pmatrix}
&
\end{align*}

This is the best basis to work with because of the convenient relations between them:\\
$$[h,e]=2e $$
$$[h,f]=-2f $$
$$[e,f]=h $$

\subsection{Irreducible representations of $\mathfrak{sl}(2,\C)$ on finite-dimensional vector spaces}
\begin{definition}An \textcolor{red}{invariant subspace} for a complex-linear representation of $sl(2,\C)$ on a finite-dimensional vector space $V$ is a complex vector subspace $U$ such that $\phi(X)U\subset U$ for all $X\in sl(2,\C)$.
\end{definition}
\begin{definition}We say that a representation on a nonzero space $V$ is \textcolor{red}{irreducible} if the only invariant subspaces are 0 and $V$ itself.
\end{definition}

Now that we have established necessary definitions, we can begin to show the irreducible representations explicitly. 

\begin{theorem}
For each integer $m\geq 1$ there exists up to equivalence a unique irreducible complex-linear representation $\pi$ of $\mathfrak{sl}(2,\C)$ on a complex vector space $V$ of dimension $m$. In $V$ thre is a basis ${v_0,v_1,\cdots,v_{m}}$ such that:
\begin{itemize}
\item $\pi(h)v_j = (m-1 - 2j)v_j$
\item $\pi(f)v_j=v_{j-1}$ with $\pi(f)v_{-1}=0$
\item $\pi(e)v_j = j(m-1+i-1)$ with $\pi(e)v_m =0$
\end{itemize}
\end{theorem}
proof:\\

CONSTRUCTION AND UNIQUENESS:\\
Let $\pi$ be a complex-linear irreducible representation of $\mathfrak{sl}(2,\C)$ on a complex vector space $V$ with dimension $m$. Since $V$ is a complex vector space and the complex numbers is a closed field, then we can find a nonzero eigenvector $v$ of $\pi(h)$ with eigenvalue $\lambda$, which by definition means that $\pi(h)v=\lambda v$. 
We can show that the vectors $\pi(e)v,\pi(e)^2v,\cdots$ are also eigenvectors of $\pi(h)$, by first using the fact that $\pi$ is a representation and therefore $\pi([h,e])=\pi(h)\pi(e)-\pi(e)\pi(h)$. Then we have:
$$\pi(h)\pi(e)v = \pi(e)\pi(h)v + \pi([h,e])v $$
Then we can use the fact that $\pi(h)v = \lambda v$ and the relations of the basis elements gives us $[h,e] =2e$, therefore we have:
$$\pi(h)\pi(e)v = \pi(e)\lambda v +\pi(2e)v $$
$$\pi(h)\textcolor{green}{\pi(e)v} = (\lambda + 2)\textcolor{green}{\pi(e)v} $$
With a similar process, we can show the rest of the vectors $\pi(e)^2, \pi(e)^3,\cdots$ will be eigenvectors with eigenvalues $\lambda +4,\lambda +6,\cdots$, respectively. It is obvious that the eigenvalues we get are distinct and therefore the eigenvectors are independent. 

By using the same method as above, using $\pi(f)$ instead of $\pi(e)$ will also give us new eigenvectors $\pi(f)v,\pi(f)^2v,\cdots$ if we use the same eigenvector v of $\pi(h)$ with eigenvalue $\lambda$, except the eigenvalues will decrease by 2 in this case. So we will have $\pi(h)\pi(f)v = (\lambda -2)\pi(f)v$. 

We can see that we are essentially moving around in different eigenspaces. This is why the basis element $e$ is called a \textcolor{red}{raising operator}, while $f$ is called a \textcolor{red}{lowering operator}. In general, these are ladder operators. This is important to address since we are essentially going to be looking at $V$ as the direct sum of the eigenspaces like so: $$V = \bigoplus_{\alpha}V_{\alpha} $$

Since $V$ is finite-dimensional, we can find a vector from $V$, let's call it $v_{0}$ for convenience, such that:
\begin{enumerate}
\item $v_{0}\neq 0$
\item $\pi(h)v_{0}=\lambda v_0$
\item $\pi(f)v_{0} = 0$
\end{enumerate}

Let us define\footnote{Knapp's book actually uses $\pi(f)$ to cycle through the vectors, but this way it makes more sense for the picture.} $v_{j} = \pi(e)^{j+1} v_{0}$. Since we know that using $\pi(e)$ increases the eigenvalue by 2, we can write it as $$\pi(h)v_j = (\lambda+2j)v_j $$
There is a minimum integer $N$ such that $\pi(e)^{N+1}v_{0} = 0$. By using the finite-dimensionality argument, $\pi(h)$ as a map of $V$ of dimension $m$ has at most $m$ distinct eigenvalues. Then the vectors $v_0,v_{1},\cdots, v_{N}$ (basically the list $v_0,ev_{0},e^2v_{0},\cdots,f^{N}v_{0}$) are independent and
\begin{itemize}
\item $\pi(h)v_j = (\lambda +2j)v_j$
\item $\pi(e)v_{j} =v_{j+1} $
\item $\pi(f)v_0 = 0$ 
\end{itemize}

Now we can show that these list of vectors span $V$ itself. For this, it is enough to show that the list of vectors
is stable (closed under)  $\pi(f)$. In fact, we can show that:
$$\pi(e)v_j = j(\lambda-j+1)v_{j-1} \text{ with } v_{-1} = 0  $$
We can prove this using induction, with the base case $j=m-1$ being $\pi(e)v_{m-1} =0 $. We assume that it is true for $j$ and we are now going to prove it for $v_{j+1}$.

$$\pi(f)v_{j+1}=\pi(f)\pi(e)v_j=\pi([f,e])v_j+\pi(e)\pi(f)v_j=$$
$$\pi(-h)v_j+\pi(e)\pi(f)v_j=-\pi(h)v_j+\pi(e)(j(-\lambda -j+1)v_{j-1})=$$
$$-(\lambda+2j)v_j - (j(-\lambda -j+1)\pi(e)v_{j-1} =-(\lambda+2j)v_j - (j(-\lambda -j+1)v_j$$
$$(-(\lambda+2j)- (j(-\lambda -j+1))v_j = (-\;ambda-j-j+j(-\lambda-j)+j)v_j= $$
$$(-\lambda-j+j(\lambda -j))v_j= (-\lambda-j)(1+j)v_j $$
\includegraphics[scale=0.5]{Screenshot_2}\\
To finish showing uniqueness, we show that $\lambda = m-1$. We have $Tr\pi(h) = Tr(\pi(e)\pi(f)-\pi(f)\pi(e))=0$ therefore we will have:
$$\sum_{j=0}^m(\lambda-2j)=0 $$
and we find that $\lambda = m$. This is really interesting, because this means the eigenvalues are integers!

EXISTENCE:\\
We define $\pi(h)$,$\pi(e)$, and $\pi(f)$ by the given the conditions given above and extend it linearly, by doing easy computation it is easy to see that:\\
$$\pi([h,e])=\pi(h)\pi(e)-\pi(e)\pi(h) $$
$$\pi([h,f])=\pi(h)\pi(f)-\pi(f)\pi(h) $$
$$\pi([e,f])=\pi(e)\pi(f)-\pi(f)\pi(e) $$

This proves that $\pi$ is a representation. To show that it is irreducible, let $U$ be a nonzero invariant subspace. Since $U$ is invariant under $\pi(h)$,$U$ is spanned by a subset of the basis vectors ${v_0,v_1,\cdots, v_{m-1}}$. Let's take a $v_k$ from the basis which is in $U$ and let's apply $\pi(f)$ several times, we will see that $V_0$ is also in $U$. Repeated application of $\pi(e)$ then shows that $U=V$. Therefore by definition $\pi$ is irreducible.
$\qed$


\subsection{Representations of $\mathfrak{sl}(2,\C)$}
\begin{definition}
Let $\phi$ be a complex-linear representation of $\mathfrak{sl}(2,\C)$ on a finite-dimensional complex vector space $V$. $V$ is \textcolor{red}{completely reducible} if we can find invariant subspaces $V_1,V_2,\cdots V_r$ of $V$ such that 
$$V=V_1\oplus V_2 \oplus \cdots V_r $$ and such that the restriction of the representation to each $V_i$ is irreducible.  
\end{definition}
\begin{theorem} 
If $V$ is a finite-dimensional complex vector space that has a complex-linear representation of $\mathfrak{sl}(2,\C)$, then $V$ is completely reducible. 
\end{theorem}

To prove this theorem we will require 4 lemmas.
\begin{lemma}
If $\pi$ is a representation of $\mathfrak{sl}(2,\C)$, then $Z=\frac{1}{2}\pi(h)^2+\pi(h)+2\pi(f)\pi(e)$ commutes with each $\pi(X)$ for $X\in \mathfrak{sl}(2,\C)$.
\end{lemma}

\begin{lemma}[Schur's Lemma]
Let $\g=\mathfrak{sl}(2,\C)$. If $\pi:\g\rightarrow ENDV$ and $\pi':\g\rightarrow EndV'$ are irreducible finite-dimensional representations and if $L:V\rightarrow V'$ is a linear map such that $L\pi(X)=\pi'(X)L$ for all $X\in \g$, then $L=0$ or $L$ is invertible. If $Z:V\rightarrow V$ is a linear map such that $Z\pi(X)=\pi(X)Z$ for all $X\in \g$, then $Z$ is scalar. 
\end{lemma}

\begin{lemma}
If $pi$ is an irreducible representation of $\mathfrak{sl}(2,\C)$ of dimension $n+1$, then the operator $Z$ of Lemma 2.3 acts as the scalar $\frac{1}{2}n^2+n$, which is not $0$ unless $\pi$ is trivial. 
\end{lemma}

\begin{lemma}
Let $\pi:\mathfrak{sl}(2,\C)\rightarrow EndV$ be a finite-dimensional representation, and let $U\subset V$ be an invariant subspace of codimension 1. Then there is a 1-dimensional invariant subspace $W$ such that $V=U\oplus W$. 
\end{lemma}
So now that we have stated the lemmas, we can prove Theorem 2.2.\\
Let $\pi$ be a representation of $\mathfrak{sl}(2,\C)$ on $M$, and let $N\neq 0$ be an invariant subspace. Put
$$V=\{Y\in EndM|Y:M\rightarrow N \text{and} Y|_{N} \text{is a scalar}\} $$
Using linear algebra we can see that $V\neq 0$. Define a linear function $\sigma:\mathfrak{sl}(2,\C)\rightarrow End(EndM)$ by 
$$\sigma(X)\gamma = \pi(X)\gamma -\gamma\pi(X) $$
for $\gamma\in EndM$ and $X\in \mathfrak{sl}(2,\C)$.\\
We can check directly that $\sigma[X,Y]=\sigma(X)\sigma(Y)-\sigma(Y)\sigma(X)$, therefore $\sigma$ is a representation of $\mathfrak{sl}(2,\C)$ on $EndM$.\\
So we can show that $V$ as a subspace of $EndM$ is an invariant subspace under $\sigma$. In fact, let $\gamma(M)\subset N$ and $\gamma|_N=\lambda1$. In the right side of the expression 
$$\sigma(X)\gamma = \pi(X)\gamma-\gamma\pi(X)$$
the first term carries $M$ to $N$ since $\gamma$ carries $M$ to $N$ and $\pi(X)$ carries $N$ to $N$, and the second term carries $M$ into $N$ since $\pi(x)$ takes $M$ to $M$ and $\gamma$ takes $M$ to $N$. Thus $\sigma(X)\gamma$ takes $M$ into $N$. On $N$, the action $\sigma(X)\gamma$ is given by 
$$\sigma(X)\gamma(n)=\pi(X)\gamma(n)-\gamma\pi(X)(n)=\lambda\pi(X)(n)-\lambda\pi(X)(n) = 0 $$
Thus $V$ is an invariant subspace. \\
So, the argument above shows that the subspace $U$ of $V$ given by 
$$U=\{\gamma\in V|\gamma=0 \text{on} N\} $$
is an invariant subspace. So, it's easy to see that $dimV/U=1$. By Lemma 2.6, $V=U\oplus W$ for a 1-dimensional invariant subspace $W=\C\gamma$, where $\gamma$ is a nonzero scalar $\lambda1$ on $N$. The invariance of $W$ means that $\sigma(X)\gamma=0$ since 1-dimensional representations are $0$. Therefore $\gamma$ commutes with $\pi(X)$ for all $X\in  \mathfrak{sl}(2,\C)$. But then the kernel of $\gamma$ is a nonzero invariant subspace of $M$. Since $\gamma$ is a nonsingular on $N$(being a nonzero scalar there), we must have $M=N\oplus ker\gamma$. 
$\qed$

\begin{corollary}
LIf $\pi$ is a complex-linear representation of $\mathfrak{sl}(2,\C)$ on a finite-dimensional complex vector space $V$, then $\pi(h)$ is diagonable, all of its eigenvalues are integers, and the multiplicity of an eigenvalue $k$ equals the multiplicity of $-k$. 
\end{corollary}
\begin{corollary}
If $\phi$ is a complex-linear representation(not necessarily an irredusible one) of $\mathfrak{sl}(2,\C)$ on a finite-dimensional complex vector space $V$, and if each vector $v\in V$ is in an finite-dimensional invariant subspace of $V$, then $V$ is the direct sum of finite-dimensional invariant subspaces on which $\mathfrak{sl}(2,\C)$ acts irreducibly.  
\end{corollary}

\end{document}